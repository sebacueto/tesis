\prefacesection{Resumen}

El desarrollo de este trabajo se encuentra entorno a la máquina de fatiga en flexión del laboratorio de tecnología mecánica, el cual tiene como objetivo avanzar hacia la operatividad de la máquina. Para esto, la metodología se dividió en 4 etapas: levantamiento de información, diseño de una estructura soportante, modelar el comportamiento de la máquina y contrastar los resultados con la información existente.

La máquina tiene problemas en su funcionamiento, mantenibilidad y disponibilidad de respuestos producto de su antigüedad. Su funcionamiento se basa en una tabla de cargas (anexo \ref{sec:anexob1}), que relaciona los contrapesos con los esfuerzos sobre probeta.

El diseño de la estructura se realizó de acero y madera, utilizando como guía la norma NCh 1198. Además, a través de MEF se simuló el comportamiento estático y modal de la estructura. De esta manera, la dimensión final de cada componente se exponen en los planos del anexo \ref{sec:planos}.

Para el modelamiento de la máquina, se utilizaron los datos obtenidos en el levantamiento de información y se resolvieron las ecuaciones de movimiento utilizando el método de energía. Así, se obtuvo el movimiento y la velocidad del centro de masa del brazo de carga. A través de esto, se calculó la fuerza realizada sobre la probeta para distintas configuraciones. La carga máxima posible, según el modelo, es de $1498.83$ [N], a una velocidad de rotación del disco $\omega_{max} =$ 25 [rad/s]. 

Estos resultados de fuerza máxima se simularon usando MEF, utilizando como límite el esfuerzo último del material. Por lo tanto, se obtuvo una nueva relación entre las combinaciones de los contrapesos y los esfuerzos en la probeta, las cuales se muestran en la propuesta \ref{sec:anexob2}.  

Con la información y los resultados obtenidos, se puede concluir que es necesaria una actualización y reparación de la máquina de fatiga, como también la construcción de la estructura para lograr que la máquina este operativa. Además existen discrepancias entre el modelo propuesto y la información existente, ante lo cual es necesario hacer un trabajo posterior que valide, refute o corrija el modelo y la tabla de cargas propuesta. 

%El desarrollo de este trabajo se encuentra entorno a la máquina de fatiga en flexión del laboratorio de tecnología mecánica, en Valparaíso. Actualmente no se encuentra operativa, por lo tanto, el objetivo de este trabajo es avanzar en la dirección que permita volver a tener operativa la máquina. Para esto, la metodología de trabajo se dividió en 4 grandes etapas: levantamiento de información, diseño de una estructura soportante para su anclaje, modelar el comportamiento de la máquina y, finalmente, contrastar los resultados con la información existente de la máquina.
%
%Esta máquina tiene una antigüedad que ronda los 60 años, lo que implica varios problemas para su mantenimiento y reparación. Diversos componentes se encuentran en desuso, alguna tecnologías están obsoletas y la información respecto a la máquina es escasa. Es por esto, que se hace necesario conocer el funcionamiento de la máquina, diseñar una nueva estructura y actualizar sus componentes. El funcionamiento actual de la máquina se basa en una tabla de carga, que relaciona la combinación de contrapesos con los esfuerzos que sufrirá la probeta a ensayar.
%
%El diseño de la estructura se realizó de acero y madera, con uniones mecánicas entre cada elemento. Para los cálculos en madera, se utiliza como guía la norma NCh 1198, la cual entrega toda la metodología de cálculo necesaria. Además, a través de elementos finitos, mediante el software Inventor, se simuló el comportamiento estático y modal. Los resultados se exponen en los planos de la estructura en el anexo \ref{sec:planos}, tanto para las dimensiones de la madera, el acero y sus uniones respectivas.
%
%Para el modelamiento del movimiento de los componentes de la máquina, se utilizaron los datos obtenidos en el levantamiento de información y se resolvieron las ecuaciones de movimiento  a través del método de energía. Así, se obtuvo el movimiento y la velocidad lineal y angular del brazo de carga respecto a su centro de masa, con lo cual fue posible calcular la fuerza realizada sobre la probeta para distintas configuraciones de contrapesos. La carga media obtenida es de $12.438$ [N] y la carga máxima posible es de $1498.83$ [N], ambas a una velocidad de rotación del disco $\omega_{max} =$ 25 [rad/s]. 
%
%Estos resultados de fuerza máxima, se simularon en elementos finitos, mediante el software ANSYS, todas las cargas hasta que la probeta alcanzó su esfuerzo último. Por lo tanto, se obtuvo una relación entre las combinaciones de los contrapesos y los esfuerzos en la probeta asociados a cada configuración, las cuales se pueden ver en la tabla de cargas propuesta, en el anexo \ref{sec:anexob2}.  
%
%Con la información y los resultados obtenidos, se puede concluir que es necesaria una actualización y reparación de la máquina de fatiga, como también la construcción de la estructura para lograr que la máquina este operativa. Además existen discrepancias entre el modelo propuesto y la información existente, ante lo cual es necesario hacer un trabajo posterior que valide, refute o corrija el modelo y la tabla de cargas propuesta.