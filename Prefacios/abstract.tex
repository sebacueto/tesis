%% Las secciones del "prefacio" inician con el comando \prefacesection{T'itulo}
%% Este tipo de secciones *no* van numeradas, pero s'i aparecen en el 'indice.
%%
%% Si quieres agregar una secci'on que no vaya n'umerada y que *tampoco*
%% aparesca en el 'indice, usa entonces el comando \chapter*{T'itulo}
%%
%% Recuerda que aqu'i ya puedes escribir acentos como: 'a, 'e, 'i, etc.
%% La letra n con tilde es: 'n.
\prefacesection{Abstract}

This work is developed around the flexural fatigue machine from mechanics technology laboratory, to move forward into its operativeness. The work methodology of this work is divided in 4 stages: get information about the machine, structural machine support design, machine modeling behaviour and contrast the previous information with results.

Nowadays, issues in the normal operavility by the machine and unavailable spares, made it unsuitable for operation. Its functioning is based on a unbalanced rotatory disc, by counterweights, to produce a specific stress.

The structural machine support was designed in wood and steel, using standard guide NCh 1198. It was simulated its static and modal behaviour, by means of FEM, to bear out the proposed design.

To characterize the machine behaviour, a dynamic model of the system was proposed. Through it, the loading arm movement and velocity are obtained. From this information, the force applied in the specimens is calculated for every configuration, achieving a relationship between $F_{max}$, $\omega_{max}$ and $\Delta m$ as a system variables.

The maximum force results, for a rotary disc velocity $\omega_{max}=25$ rad/s, are simulated using FEM to relate stress and strain state of the specimen with each configuration. Futhermore, using the relationship between each variables, it is founded the unbalance $\Delta m_y$ for yielding stress and $\Delta m_u$ for ultimate tensile strenght in the specimen. Lastly, fatigue life ($N_f$) is calculated for each configuration.

Thus, was possible to conclude that an update and maintenance was needed, as well as build the structural machine support to return the correct operation of the machine. On the other hand, there was discrepancies between the proposed model and the previus information. Hence, experimental validation are necessary as a later work.