%% Las secciones del "prefacio" inician con el comando \prefacesection{T'itulo}
%% Este tipo de secciones *no* van numeradas, pero s'i aparecen en el 'indice.
%%
%% Si quieres agregar una secci'on que no vaya n'umerada y que *tampoco*
%% aparesca en el 'indice, usa entonces el comando \chapter*{T'itulo}
%%
%% Recuerda que aqu'i ya puedes escribir acentos como: 'a, 'e, 'i, etc.
%% La letra n con tilde es: 'n.

\prefacesection{Agradecimientos}
Es difícil escribir y sintetizar en un par de párrafos los agradecimientos a un grupo de personas que han aparecido a lo largo de estos 8 años. Mis agradecimientos van principalmente hacia aquellas personas que me han marcado profundamente en este proceso, pero también hacia aquellas personas que de alguna o otra forma incidieron en mi paso por la universidad, incluso a aquellos que no conocí ni saludé. Son estos dos conjuntos de personas quienes agradezco por ser parte de este devenir, quienes en algún punto influyeron en mis tiempos, mis decisiones y todo aquello sobre lo que no tuve control, pero que finalmente, se acoplaron para poder dar cierre a este ciclo de mi vida.

Quiero agradecer a mi familia, a mi mamá y papá por apoyarme durante toda esta etapa, por permitirme el privilegio de dedicarme exclusivamente a los estudios y a otras cosas, inclusive. Por brindarme todas las herramientas necesarias, desde que nací hasta hoy, entregándome su tiempo, disposición y amor sin esperar retribución alguna, por retarme y alentarme cuando fuese necesario, pero por sobretodo por creer en mí y en mis capacidades. A mi abuela Rosa y mi abuelo Hernán, quienes llegaron a ser unos segundos padres al consintirme, formarme e impregnarme de su amor por los libros, la música, los relatos y su cariño.

Por otro lado, también agradezco a mis amigas Susi, Camila y Catalina, a mis amigos Patricio, Marcelo, Pablo Campos, Pablo Kohler, Cristián, Sebastián y Pablo Cárdenas, quienes conozco desde mis primeros años en la universidad o antes, que me apoyaron en los momentos difíciles y me felicitaron en los momentos alegres. Les agradezco su capacidad de discusión, de poder opinar libremente y la idea de siempre buscar construirnos como mejores personas, por su infaltable cariño y afecto que hizo más ameno mi paso por la universidad, en especial a los tres últimos quienes me ayudaron a desarrollar esta tesis con sus conocimientos y el abrirse a escucharme. Por otro lado, quiero agradecer a Laura, quien fue mi amiga y compañera, por su inconmensurable amor, dedicación y paciencia, por su capacidad de creer en mí, aún en los momentos más oscuros. No fueron sino las risas, la compañía y su sabiduría lo que me permitieron llegar hasta donde estoy.

A mis compañeros de banda, con quienes durante 8 años ensayamos, compartimos y nos enseñamos alrededor de la música. La excusa de juntarse a tocar instrumentos de forma coordinada desencadenó en una amistad y un espacio de aprendizaje profundo que sobrepasa las dimensiones de la banda misma. Sin duda, cada ensayo y cada creación fueron un escape de la universidad, permitiéndonos dialogar y expresarnos a través de nuestra música, pero también al abrirnos y exponer nuestros sentimientos en la conversación.

Al profesor Alejandro Pacheco, con quien tuve la suerte de desarrollar esta tesis, contar con su apoyo e infinita paciencia. La dedicación que tuvo al destinar de su tiempo para enseñarme, responder mis dudas y corregirme cada vez que me equivocaba fueron una constante luz en este largo proceso.  Sin duda su empuje a estudiar, comprender y aplicar lo aprendido me dejo una marca importante en mi formación que espero lograr plasmar en mi futuro como ingeniero. 

Finalmente, me gustaría agradecer a don Vicente Álvarez, apoyo académico del laboratorio de tecnología mecánica, quien siempre estuvo dispuesto a ayudarme, enseñarme y abrir la máquina un sin número de veces, sus palabras de ánimo y grata disposición fueron un gran apoyo y facilitó en gran medida el desarrollo de este trabajo.


