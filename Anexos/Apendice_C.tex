%Apendice C
%
\chapter{Solver, función y scripts para la resolución del modelo del sistema en MATLAB}
\label{ch:anexo_c}

\section{Solución para un caso particular}
\label{sec:sol_part}

\subsection{Función}
\label{sec:fun_part}

\begin{lstlisting}

function DERVAR = machfun(t, XVAR)

% Valores de entrada

% Configuracion n 180
m1= 83.1959e-3;      % m_1 y m_2 corresponden a los contrapesos.
m2= 6.8541e-3;       % m_1 > m_2 [kg]

% Tipo de probeta
E_p= 2e11;           % Mod. de Young del material de la probeta [Pa]
I_p= 3.309e-10;      % Segundo momento de area de la probeta [m4].

% Parametros del sistema
R_d = 0.1120 ;       % Radio disco de aluminio [m]
Md = 19.2029;        % Masa disco y conjunto [kg]
E_ac = 2e11;         % Modulo de Young acero [Pa]
I_b1 = 5.6718e-8;    % Segundo momento de area brazo 1 [m4]
L_b1 = 33.3e-2;      % Largo brazo 1 [m]
l = 0.2005;          % Largo brazo 2 entre P y Q [m]
l1 = 0.110337;       % Largo brazo 2 entre P y centro de masa Gb [m]
l2 = 0.090163;       % Largo brazo 2 entre centro de masa Gb y Q [m]
Mb = 2.3050;         % Masa brazo 2 [kg]
Igb = 8.32726e-4;    % Momento de inercia brazo 2 [kg*m2]
L_p = 0.0724;        % Largo de probeta [m]
g = 9.806;           % Gravedad [m/s2]

k1 = 98890.45;       % Coef. de elasticidad de las barras de acero
k2 = 1.798e6;        % Coef. de elasticidad de la probeta de acero

% Calculos
Ma = m1 + m2 + Md;
ega = (R_d*(m1-m2))/(Ma);
c1 = 100;
c2 = 100;

% Funcion de aceleracion del disco
[angle,anglep,anglepp] = vinput(t);

% Asignacion de variables
Y = XVAR(1);
Yp = XVAR(2);
B = XVAR(3);
Bp = XVAR(4);

DERVAR(1) = Yp;

Ypp = (Ma*ega*(anglep*anglep*sin(angle) - anglepp*cos(angle))*(Igb + 2*Ma*l1*l1) - c1*(Yp - l1*Bp)*Igb - c2*(Yp + l2*Bp)*(Igb + Ma*l1*l1 + Ma*l1*l2) - k1*(Y - l1*B)*Igb - k2*(Y + l2*B)*(Igb + Ma*l1*l1 + Ma*l1*l2) - g*((Ma + Mb)*(Igb + Ma*l1*l1) - Ma*Ma*l1*l1))/((Ma + Mb)*(Igb + Ma*l1*l1) - Ma*Ma*l1*l1);

DERVAR(2) = Ypp;

DERVAR(3) = Bp;

Bpp = (Ma*ega*(anglep*anglep*sin(angle) - anglepp*cos(angle))*(2*Ma*l1 + l1*Mb) + c1*(Yp - l1*Bp)*Mb*l1 - c2*(Yp + l2*Bp)*(Ma*l1 + l2*(Ma + Mb)) + k1*(Y - l1*B)*Mb*l1 - k2*(Y + l2*B)*(Ma*l1 + l2*(Ma + Mb)))/((Ma + Mb)*(Igb + Ma*l1*l1) - Ma*Ma*l1*l1);

DERVAR(4) = Bpp;

DERVAR = DERVAR';

\end{lstlisting}


\subsection{Solver}
\label{sec:solver_part}

\begin{lstlisting}

% Tolerancia relavita y absoluta
opts = odeset('RelTol',1e-8,'AbsTol',1e-8);		

% Solver

[tp,x]=ode45(machfun,[0,8],[0;0;0;0],opts);


% Valores auxiliares
l2 = 0.090163;				
k2 = 1.798e6;			

% Fuerza sobre la probeta
F2 = (x(:,1)+l2*x(:,3))*k2;						


\end{lstlisting}

\section{Función de la aceleración, velocidad y posición angular del disco}
\label{sec:phi_fun}

\begin{lstlisting}
function [t,tp,tpp] = vinput(time)

Ti = 2;
W_rpm = 1500;
WMAX = W_rpm*2*pi/60;
T = 2.5;

for i=1:length(time)
    
    if(time(i) < Ti)   
    	t(i) = 0;
    	tp(i)=0;
    	tpp(i)=0;
    
    elseif (time(i) < (Ti + T)) 
    	t(i) = (WMAX/(6*T^2))*((time(i)^3) - 3*Ti*(time(i)^2) + 3*time(i)*(Ti^2) - Ti^3);
    	tp(i) = (WMAX/(2*T^2))*((time(i)^2) - 2*Ti*time(i) + Ti^2);
    	tpp(i) = (WMAX/(T^2))*(time(i) - Ti);
    
    elseif (time(i) < (Ti + 2*T))
    	t(i) = (WMAX/T)*(time(i)^2 - 2*Ti*time(i) - 3*T*time(i) + (1/(2*T))*(4*time(i)*T^2 - time(i)*Ti^2 - ((time(i)^3)/3) + Ti*time(i)^2)) + ((2*WMAX*T^3 + 6*WMAX*Ti*T^2 + 6*WMAX*T*Ti^2 + WMAX*Ti^3)/(6*T^2));  
    	tp(i) = (WMAX/T)*(2*time(i) - 2*Ti - 3*T + (1/(2*T))*(4*T^2 - Ti^2 - (time(i)^2) + 2*Ti*time(i)));
    	tpp(i) = (WMAX/T)*(2 - ((time(i) - Ti)/T));
    
    else
    	t(i) = WMAX*time(i) - WMAX*(T + Ti);
    	tp(i) = WMAX;
    	tpp(i) = 0;
    
    end
end
\end{lstlisting}

\section{Script para cargas máxima, media y alternante}
\label{sec:sol_gen}
A la matriz con cada combinación de la masa de los contrapesos, escritas en el orden de la tabla de cargas, se denomina \textit{M} y está contenida en el script \textit{''MassT''}
\begin{lstlisting}
MassT;
F1500 = zeros(length(M),1);
dm = zeros(length(M),1);
Me = zeros(length(M),1);
Fmax1500 = zeros(length(M),1);
Fm1500 = zeros(length(M),1);

for i = 1:1:length(M)
    i	% Contador

    m1 = M(i,1); % Combinaciones contenidas en la matriz M
    m2 = M(i,2);
    
    
    opts = odeset('RelTol',1e-8,'AbsTol',1e-8);

    [tp,x]=ode45((t,XVAR)machfun_test(t,XVAR,m1,m2), [0,8],[0;0;0;0],opts);

    laux = length(tp);
        
    l2 = 0.090163;
    k2 = 1.798e6;
    l1 = 0.110337;
        
    dy = x(:,1) + l2*x(:,3);
    Fp = dy*k2;
    Faux = Fp(round(laux/2):laux,:); 
    Fabs = abs(max(Faux) - min(Faux));
    Fm = (abs(max(Faux) + min(Faux)))/2;
    fmax = abs(max(Faux));
    fmin = abs(min(Faux));
        if fmax > fmin
            Fmax1500(i) = fmax;
        else
            Fmax1500(i) = fmin;
        end
    F1500(i) = Fabs/2;
    Fm1500(i) = Fm;
    dm(i) = m1-m2;
    
    Rd = 0.1120 ;  
    Md = 19.2029;   
    ega = (Rd*(m1-m2))/(m1+m2+Md);
    Ma = Md + m1 + m2;
    Me(i) = ega*Ma;
    
    save('resultados_tabla_1500.mat','F1500','dm','Me','Fmax1500','Fm1500');
    
end
\end{lstlisting}