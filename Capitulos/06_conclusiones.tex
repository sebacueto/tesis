\chapter{Conclusiones}

\subsubsection{Respecto al levantamiento de información:}

Gran parte de la información recopilada y de las mediciones realizadas se llevaron a cabo en las condiciones necesarias, otorgando confiabilidad a los datos. Sin embargo, ciertos elementos no fue posible medirlos o determinar sus características de manera certera ante la imposibilidad de desarmar los componentes de la máquina. En específico, la masa y las dimensiones del disco desbalanceado y los elementos que lo acompañan como la polea, el eje, los agarres laterales y las barras de transmisión de la carga. Sin duda esta falta de precisión afecta los cálculos que se realizaron y los resultados obtenidos. 

La robustez de la estructura de la máquina, su mecanismo de funcionamiento y la capacidad de medir fatiga en flexión y torsión, hace que su operatividad sea necesaria para su uso en ensayos de fatiga como laboratorio de una asignatura. Además, la abundancia en distintos fenómenos mecánicos y la facilidad de visualizarlos, puede ser de ayuda en la comprensión de variados conceptos en sus respectivas asignaturas. Así mismo, la operación abre la posibilidad de modificar la máquina para aumentar su flexibilidad en la configuración de los ensayos, añadir elementos de medición y por sobre todo actualizar sus componentes.

En base a esto último, para garantizar un correcto funcionamiento de la máquina, urge reemplazar las poleas y su respectiva correa por medidas que sean estándar en la actualidad. Esto porque, el material y las dimensiones de la actual correa son difíciles de encontrar en el mercado por su obsolescencia, debiendo recurrir a alternativas que no son óptimas ni poco efectivas. 

\subsubsection{Sobre la estructura y su diseño}

La utilización de la madera como material principal de la estructura soportante de la máquina es una opción viable debido a su capacidad de amortiguar las vibraciones y su resistencia al a fatiga, como se señaló en la sección REFaAntecedentes. Por otro lado, el construir una estructura que pueda modificada y desplazada de lugar, le otorga una mayor flexibilidad en su funcionalidad, respecto a la fundación original de un bloque de concreto, lo que evitaría que vuelva a quedar sin anclar frente a posibles modificaciones futuras en el laboratorio de tecnología mecánica. 

A lo largo de todo el diseño de la estructura se utilizó la norma NCh 1198 y el apoyo de textos de construcción en madera, para otorgar fiabilidad a la resistencia de la estructura y utilizar diseños y mecanismos de unión propios de la madera y que, por sus características ya señaladas, tienen una lógica y aproximación distinta. La norma, al ser un marco y un instructivo a la hora de calcular las cargas, utilizado actualmente para la construcción de distintas estructuras en madera, responde a todas las necesidades asociadas al diseño de la estructura soportante de la máquina. \textcolor{red}{revisar o añadir más cosas}.

Los factores de seguridad obtenidos en la madera, al ser tan altos, dan un margen para la fatiga que sufrirá el material a lo largo de su uso. En el caso de COMPONENTE, su factor de seguridad de FS = CUANTO, es el más bajo obtenido en el diseño, siendo bastante alto para un diseño convencional.

\subsubsection{Modelo del sistema vibratorio}

Como se señaló en la sección REFaResultados, los resultados que se obtuvieron del modelo no coinciden con la forma en que se ordenan las cargas 




\section{Trabajo futuro}
El trabajo actual abre espacio a distintos análisis posibles en torno a la misma máquina de fatiga, tanto en diseño, construcción, corroboración empírica, análisis y simulación. Por consiguiente, se detallarán a continuación:
\begin{itemize}
	\item Caracterización, reparación y actualización de la máquina de fatiga. Para llevar a cabo esto, es necesario poder conocer en detalle cada una de sus partes y por lo tanto se requiere la necesidad de desarmarla. Tanto su reparación como actualización van de la mano de lo anterior, además de realizar posibles modificaciones a la estructura y a su sistema motriz.
	\item Construcción y diseño de amortiguamiento de la estructura. Construir el diseño de la estructura propuesto para poder anclar la máquina de fatiga. Además, a partir de las pruebas realizadas una vez anclada, se debe estimar si es necesaria la colocación de amortiguadores para disipar las vibraciones producto de la operación de la máquina de fatiga, sin que afecte su entorno.
	\item Realizar un modelo de vibración de la máquina de fatiga en su configuración en torsión, para obtener una tabla de cargas y sus respectivos esfuerzos.
	\item Corroborar si el modelo y las simulaciones realizadas son correctas con la realidad, realizando las mediciones correspondientes en la probeta, el brazo de carga y el resto de los elementos involucrados.
	\item Una vez que la máquina se encuentre operativa, es posible realizar ensayos de fatiga tanto de acero, como de distintos materiales o de probetas con secciones distintas, permitiendo comparar las curvas $S$-$N$ obtenidos a partir de la máquina de fatiga con los datos de otras formas de medición.
\end{itemize}










