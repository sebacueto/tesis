\chapter{Análisis y Resultados}

\section{Levantamiento de información}
Por medio del levantamiento de información es posible hacer un pequeño análisis del estado actual y los datos obtenidos. Por un lado, la estructura actual responde a la necesidad de colocar la máquina en una ubicación transitoria, debido a los trabajos realizados en el piso del laboratorio durante el año 2012. Por lo mismo, no está diseñada para la operación de la máquina bajo ningún contexto por los peligros que conlleva. Aún cuando esta podría ser modificada, se desconocen las características y propiedades de la especie maderera con la que fue fabricada, lo que obliga a diseñar y construir un nuevo soporte para la máquina. 

Por otra parte, la información obtenida de la máquina de fatiga da cuenta de tres puntos importantes. La antigüedad de sus componentes y la tecnología utilizada afecta directamente en su mantenimiento ante la dificultad de encontrar piezas de repuesto, debido a que las dimensiones de sus componentes, como la polea y la correa, se encuentran fuera de catálogo o de las dimensiones de fabricación de los proveedores. También la obsolescencia de la tecnología afecta su mantenimiento, siendo difícil poder encontrar no sólo las piezas de repuesto, sino que también personas que estén técnicamente calificadas. En segundo lugar, la máquina fue fabricada con estándares o líneas de desarrollo propias de la época, como sería esperable, las cuales para el caso de este trabajo no evolucionaron en la misma dirección de que los estándares actuales de ensayo a fatiga, dificultando la comparación de los resultados obtenidos. Finalmente, la robustez del diseño 

Confiabilidad de los datos