\chapter{Conclusiones}

\subsubsection{Respecto al levantamiento de información:}

Gran parte de la información recopilada y de las mediciones realizadas se llevaron a cabo en las condiciones necesarias, otorgando confiabilidad a los datos. Sin embargo, ciertos elementos no fue posible medirlos o determinar sus características de manera certera ante la imposibilidad de desarmar los componentes de la máquina. En específico, la masa y las dimensiones del disco desbalanceado y los elementos que lo acompañan como la polea, el eje, los agarres laterales y las barras de transmisión de la carga. Sin duda esta falta de precisión afecta los cálculos que se realizaron y los resultados obtenidos. 

La robustez de la estructura de la máquina, su mecanismo de funcionamiento y la capacidad de medir fatiga en flexión y torsión, hace que su operatividad sea necesaria para su uso en ensayos de fatiga como laboratorio de una asignatura. Además, la abundancia en distintos fenómenos mecánicos y la facilidad de visualizarlos, puede ser de ayuda en la comprensión de variados conceptos en sus respectivas asignaturas. Así mismo, la operación abre la posibilidad de modificar la máquina para aumentar su flexibilidad en la configuración de los ensayos, añadir elementos de medición y por sobre todo actualizar sus componentes.

En base a esto último, para garantizar un correcto funcionamiento de la máquina, urge reemplazar las poleas y su respectiva correa por medidas que sean estándar en la actualidad. Esto porque, el material y las dimensiones de la actual correa son difíciles de encontrar en el mercado por su obsolescencia, debiendo recurrir a alternativas que no son óptimas ni poco efectivas. 

\subsubsection{Sobre la estructura y su diseño}

La utilización de la madera como material principal de la estructura soportante de la máquina es una opción viable debido a su capacidad de amortiguar las vibraciones y su resistencia al a fatiga, como se señaló en la sección REFaAntecedentes. Por otro lado, el construir una estructura que pueda modificada y desplazada de lugar, le otorga una mayor flexibilidad en su funcionalidad, respecto a la fundación original de un bloque de concreto, lo que evitaría que vuelva a quedar sin anclar frente a posibles modificaciones futuras en el laboratorio de tecnología mecánica. 

A lo largo de todo el diseño de la estructura se utilizó la norma NCh 1198 y el apoyo de textos de construcción en madera, para otorgar fiabilidad a la resistencia de la estructura y utilizar diseños y mecanismos de unión propios de la madera y que, por sus características ya señaladas, tienen una lógica y aproximación distinta. La norma, al ser un marco y un instructivo a la hora de calcular las cargas, utilizado actualmente para la construcción de distintas estructuras en madera, responde a todas las necesidades asociadas al diseño de la estructura soportante de la máquina. \textcolor{red}{revisar o añadir más cosas}.

Los factores de seguridad obtenidos en la madera, al ser tan altos, dan un margen para la fatiga que sufrirá el material a lo largo de su uso. En el caso de COMPONENTE, su factor de seguridad de FS = CUANTO, es el más bajo obtenido en el diseño, siendo bastante alto para un diseño convencional.

\subsubsection{Modelo del sistema vibratorio}
