\chapter{Metodología}
\section{Levantamiento de información}
El departamento de ingeniería mecánica en el Campus Casa Central, posee una máquina de fatiga (MF) uniaxial en flexión en el laboratorio de tecnología de materiales. \textcolor{red}{Revisar nombre del lab} La se encuentra en la universidad hace más de 50 años, sin saber su fecha exacta de adquisición, y su fabricante es la compañía \textit{Amsler \& co.}\textcolor{red}{Revisar nombre}, la cual fue adquirida por la actual compañía \textit{Zwell} \textcolor{red}{revisar}. La medición de fatiga es realizada a través del método de \textit{esfuerzo-vida}, utilizando la configuración de \textit{rotating bending}, ambos descritos en el capítulo anterior. La información existente sobre la misma es escasa, principalmente por su antigüedad, la perdida de sus documentos y tecnología obsoleta. Por lo mismo, parte del trabajo de esta memoria se centra en lograr rescatar información y su posterior comprensión para lograr la operatividad de MF.

%Para poder llevar a cabo este trabajo, fue necesario obtener información sobre la máquina de fatiga, su estructura, su historia, funcionamiento y el estado actual.
\subsection{Estado actual}
Actualmente, la máquina no se encuentra anclada al piso o una estructura que impida su movimiento, estando apoyada sobre dos listones de madera, que a su vez, están sobre una mesa de madera que tampoco se encuentra empotrada o anclada al piso. Por consiguiente, la máquina al ser utilizada comienza a vibrar y a desplazarse lateralmente. Por lo tanto, por motivos de seguridad no es posible realizar ensayos en ella. 

Por otro lado, la máquina no contiene modificaciones mayores, ya sea en su estructura, electrónica o su drive section \textcolor{red}{revisar}. Según la información recopilada, su única modificación se encuentra en el contador de revoluciones, el cual fue reemplazado de uno mecánico a otro electrónico. 


\subsection{Mediciones}

\subsection{Funcionamiento}
\subsubsection{Mecánica}
\subsubsection{Electrónica}
\subsubsection{Recolección de datos}


\section{Diseño de estructura}
\subsection{Diseño en soporte de acero}

\subsection{Diseño en madera}

\subsection{Cálculo de cargas en estructura de madera}

\subsection{Uniones}
\subsubsection{Acero - madera}
\subsubsection{Madera - Madera}
\subparagraph{Pernos}
\subparagraph{Tirafondos}
\subsubsection{Herrajes y conectores}

\subsection{Simulaciones}
\subsubsection{Estática}
\subsubsection{Modal}


\section{Modelo del sistema de funcionamiento}
\subsection{Diagrama del sistema}

\subsection{Modelo del sistema}
\subsubsection{Simplificaciones}
\subsubsection{Modelo del disco desbalanceado}
\subsubsection{Cálculo y obtención de valores del sistema}

\subsection{Ecuaciones de movimiento de la máquina de fatiga}

\subsection{Carga sobre la probeta}

\section{Análisis de información existente}
\subsection{Simulación de cargas}