\chapter{Introducción}

Las prótesis y la impresion 3D son tópicos que han ido tomando importancia durante la última década. La impresión 3D, como tecnología, ha crecido fuertemente desde la aparición de las impresoras de escritorio, las cuales permiten a cualquier persona sin conocimientos específicos de ingeniería, poder imprimir piezas u objetos diseñados por el propio usuario sin tener que contar con equipo especializado. Un ejemplo de este rápido crecimiento se puede notar en una noticia del periódico The Economist de hace una década: 

\begin{quote}
\textit{If you really want to impress your friends with high-tech wizardry in 2008 then consider shopping for a three-dimensional printer. (``A Whole New Dimension'', 2007)}
\end{quote}


Esta transición de una tecnología restringida a la industria especializada hacia lo privado o industrias de pequeña escala, llevó a un desarrollo y aplicación de la impresión 3D más disgregada, independiente de las principales marcas fabricantes de impresoras, permitiendo el surgimiento de comunidades, startups o incluso iniciativas universitarias centradas en la investigación o desarrollo de aplicaciones y mejoras de esta tecnología. Como consecuencia de esto se crearon nuevas marcas de impresión 3D como MakerBot, surgieron comunidades de libre acceso como RepRap y también diversas inciativas desarrollaron prótesis, principalmente de brazo, impresas en 3D.\\

Así, la impresión 3D surge como una solución a las diversas problemáticas que enfrentaba el desarrollo de las prótesis, entre ellas, la reducción de costos y la adaptabilidad de cada paciente. De esta forma, distintos diseños, desarrollos y alternativas surgen desde distintos lugares y con objetivos distintos. Por un lado, existe un desarrollo de libre acceso destinado a solucionar de manera rápida y autónoma las problemáticas de las personas discapacitadas, buscando que cada usuario pueda modificar e imprimir sus prótesis. Y por otro, empresas han buscado crear o mejorar diseños existentes con el objetivo de poder entregar un producto que se adapte mejor a cada paciente y situación, sin los grandes costos que implica la compra de una prótesis tradicional.\\

Bajo este contexto, es que el presente trabajo de título está enmarcado en el proyecto del análisis de una prótesis transtibial impresa en 3D. A través de este proyecto se busca conocer el comportamiento mecánico de la prótesis y, de esta forma, lograr predecir su vida útil.

