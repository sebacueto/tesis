\chapter{Introducción}

Las prótesis y la impresion 3D son tópicos que han ido tomando importancia durante la última década. La impresión 3D, como tecnología, ha crecido fuertemente desde la aparición de las impresoras de escritorio, las cuales permiten a cualquier persona sin conocimientos específicos de ingeniería, poder imprimir piezas u objetos diseñados por el propio usuario sin tener que contar con equipo especializado. Un ejemplo de este rápido crecimiento se puede notar en una noticia del periódico The Economist de hace una década: 

\begin{quote}
\textit{If you really want to impress your friends with high-tech wizardry in 2008 then consider shopping for a three-dimensional printer. (``A Whole New Dimension'', 2007)}
\end{quote}


Esta transición de una tecnología restringida a la industria especializada hacia lo privado o industrias de pequeña escala, llevó a un desarrollo y aplicación de la impresión 3D más disgregada, independiente de las principales marcas fabricantes de impresoras, permitiendo el surgimiento de comunidades, startups o incluso iniciativas universitarias centradas en la investigación o desarrollo de aplicaciones y mejoras de esta tecnología. Como consecuencia de esto se crearon nuevas marcas de impresión 3D como MakerBot, surgieron comunidades de libre acceso como RepRap y también diversas inciativas desarrollaron prótesis, principalmente de brazo, impresas en 3D.

Así, la impresión 3D surge como una solución a las diversas problemáticas que enfrentaba el desarrollo de las prótesis, entre ellas, la reducción de costos y la adaptabilidad de cada paciente. De esta forma, distintos diseños, desarrollos y alternativas surgen desde distintos lugares y con objetivos distintos. Por un lado, existe un desarrollo de libre acceso destinado a solucionar de manera rápida y autónoma las problemáticas de las personas con discapacidad motora, buscando que cada usuario pueda modificar e imprimir sus prótesis. Y por otro, empresas han buscado crear o mejorar diseños existentes con el objetivo de poder entregar un producto que se adapte mejor a cada paciente y situación, sin los grandes costos que implica la compra de una prótesis tradicional.

Dependiendo del tipo de prótesis, estarán sometidas a distintas cargas estáticas y fluctuantes y tendrán un uso más reiterativo o puntual. En consecuencia, surge la necesidad de tener información respecto al comportamiento mecánico del material manufacturado bajo condiciones específicas como lo es la impresión 3D por deposición fundida (FDM, por sus siglas en inglés). Para el plástico ABS, existen distintos estudios respecto a sus propiedades a tensión y compresión bajo distintas configuraciones de carácter estático, sin embargo, la información disponible sobre su comportamiento bajo cargas dinámicas es bastante escaso, dificultando la predicción de su vida útil y, a su vez, la confiabilidad que tendrá durante su uso.

Así, las cargas variables generan un daño en las estructuras al provocar micro-grietas que se expanden con el tiempo y que reducen las propiedades mecánicas del material. Este daño provocado por una o varias cargas variables y repetitivas en el tiempo es llamada fatiga y la resistencia a la fatiga es la capacidad de un material de soportar este daño. Para estudiar las propiedades de un material o elemento frente a la fatiga, se deben realizar ensayos que logren dar información relevante para los casos de estudio, como en este caso son las prótesis, existiendo distintas tipos de máquinas, normas y tecnologías que se adaptan a las distintas necesidades.

De este modo, este trabajo se enmarca en el proyecto de análisis de una prótesis transtibial fabricada a través de FDM con material ABS. Con este proyecto se busca conocer el comportamiento mecánico de la prótesis bajo cargas estáticas y fluctuantes, como lo son el estar de pie y caminar, para predecir la vida útil del diseño. No obstante, como se señaló anteriormente, la información existente sobre las propiedades de fatiga del material son bastante escasas, lo que nos lleva a la necesidad de generar información útil que pueda servir como datos de entrada a las simulaciones computacionales de la prótesis.

Bajo este contexto es que el presente trabajo de título busca diseñar las condiciones necesarias para la puesta en marcha de la máquina de fatiga a flexión existente en el laboratorio de tecnología mecánica del departamento de ingeniería mecánica en el campus Casa Central. Para esto, se hará un nuevo diseño para la estructura soportante que logre anclar la máquina y resistir su operación. Además se levantará información respecto a su funcionamiento, para posteriormente realizar un modelo que busque describir y predecir el movimiento de la máquina y sus componentes para comprender la carga a la que se somete la probeta. Por último, se verificarán las cargas obtenidas simulando mediante un software de elementos finitos los esfuerzos asociados a cada configuración de la máquina, para en último término, contrastar los datos conseguidos con la información existente previamente en el laboratorio.

